\documentclass{beamer}
\usetheme{Warsaw}
\usepackage[ngerman]{babel}
\usepackage[T1]{fontenc}
\usepackage{lmodern}
\usepackage{amsmath}
\usepackage{amsthm}
\usepackage{amsfonts}
\usepackage{amssymb}
\usepackage{color}
\usepackage{dsfont}
\usepackage{bibgerm}
\usepackage{hyperref}
\usepackage{graphicx}
\usepackage{ragged2e}
\usepackage{tikz}
\usepackage{verbatim}
\usepackage[active,tightpage]{preview}
\PreviewEnvironment{tikzpicture}
\setlength\PreviewBorder{1pt}
\usetikzlibrary{trees}

\setbeamertemplate{theorems}[]
\newtheorem{thmS}{Satz}
\newtheorem{thmP}{Proposition}
\newtheorem{thmL}{Lemma}
\newtheorem{thmK}{Korollar}
\newtheorem{thmD}{Definition}
\newtheorem{thmBe}{Bemerkung}
\newtheorem{thmA}{Annahme}
\newtheorem{thmB}{Beispiel}

\title{Dynamic Mechanism-Design}
\author{Michael F\"ug und Philip Zilke}
\date{28. April 2016}

\begin{document}
\maketitle

\frame{\tableofcontents}

%Kapitel:
\section{Vom statischen zum dynamischen Mechanismus Design}

\begin{frame}
\frametitle{Aufbau des Vortrages}
\justifying
Die Hauptans\"atze:
\begin{itemize}
    \item \textcolor{blue}{Dynamische private Informationen} und statische Allokationen
    \item \textcolor{blue}{Dynamische Allokationen} und statische private Informationen
\end{itemize}

\end{frame}

\section{Dynamische private Informationen}
\begin{frame}
  \frametitle{Modelrahmen}
  \justifying
  Modellrahmen ist Zwei-Personen Spiel:
  \begin{itemize}
    \item \textcolor{blue}{Verk"aufer}
    \begin{itemize}
      \item Verkauft unteilbares Gut
      \item Legt Mechanismus $\Gamma$ fest
    \end{itemize}
    \item \textcolor{blue}{K"aufer}
    \begin{itemize}
      \item Bewertung des Gutens $\theta$ zun"achst unbekannt
      \item Erh"alt Signal $\tau$, welches mit $\theta$ korreliert ist
      \item Wenn Mechanismus aktzeptiert wird, dann wird $\tau$ berichtet
      \item $\theta$ ist erst nach Aktzeptieren des Mechanismus $\Gamma$ und wird danach berichtet
    \end{itemize}
  \end{itemize}
\end{frame}

\begin{frame}
  \frametitle{Mathemtische Modellierung}
  \justifying
  Sei im Folgenden f"ur das Signal $\tau$
  \begin{itemize}
    \item Kommulierte Verteilung $G(\tau)$
    \item Positive Dichte $g(\tau)$
    \item Tr"agermenge $[ \b{\tau} , \bar{\tau} ]$
  \end{itemize}
  Sei im Folgenden f"ur die Bewertung $\theta$
  \begin{itemize}
    \item Kommulierte Verteilung $F(\theta \mid \tau)$
    \item Korrespondierende Dichte $f(\theta \mid \tau)$
    \item Tr"agermenge $[ \b{\theta} , \bar{\theta} ]$ mit $0 \leq \b{\theta} < \bar{\theta} $ f"ur alle $\tau \in [ \b{\tau} , \bar{\tau} ]$
  \end{itemize}
\end{frame}

\begin{frame}
  \frametitle{Annahmen}
  \justifying
  Wir setzen im Folgenden voraus, dass
  \begin{itemize}
    \item Tr"agermenge von $F(\theta \mid \tau)$ ist $[ \b{\theta} , \bar{\theta} ]$ f"ur alle
    $\tau \in [ \b{\tau} , \bar{\tau} ], 0 \leq \b{\theta} < \bar{\theta}$
    \item Tr"agermenge von $F(\theta \mid \tau)$ unabh"angig von $\tau$
    \item $f(\theta \mid \tau) > 0$ f"ur alle $\tau \in [ \b{\tau} , \bar{\tau} ]$
    und $\theta \in [ \b{\theta} , \bar{\theta} ]$
    \item $F(\theta \mid \tau)$ und $f(\theta \mid \tau)$ sind stetig differenzierbar in $\tau$
    \item F"ur die Familie $F( \cdot \mid \tau)$ mit $\tau \in [ \b{\tau} , \bar{\tau} ]$ gilt
    \begin{equation}
      \tag{FOSD}
      \delta F(\theta \mid \tau)/ \delta \tau < 0 \ f"ur \ alle \ \theta \in [ \b{\theta} , \bar{\theta} ]
    \end{equation}
  \end{itemize}
\end{frame}

\subsection{Dynamischer direkter Mechanismus}
\begin{frame}
  \frametitle{Dynamischer direkter Mechanismus}
  \justifying
  \begin{thmD}
    Ein (dynamischer) \textcolor{blue}{direkter Mechanismus} besteht aus den beiden Funktionen
    \begin{equation*}
      q:[ \b{\tau} , \bar{\tau} ] \times [ \b{\theta} , \bar{\theta} ] \rightarrow [0,1]
      \quad \text{und} \quad
      t:[ \b{\tau} , \bar{\tau} ] \times [ \b{\theta} , \bar{\theta} ] \rightarrow \mathbb{R}.
    \end{equation*}
  \end{thmD}
  Zwei wesentliche Haupt"anderungen:
  \begin{enumerate}
    \item Verwende nun das kartesische Produkt $[ \b{\tau} , \bar{\tau} ] \times [ \b{\theta} , \bar{\theta} ]$
    \item Zwei Report-Ebenen: $\tau$ und $\theta$ (dynamisch hinsichtlich Zeit)
  \end{enumerate}
\end{frame}

\begin{frame}
  \frametitle{Ein-Personen-Entscheidungsproblem}
  \justifying
  \justifying
  Optimale Entscheidung ist das Paar $\sigma = (\sigma_{1}, \sigma_{2})$ mit
  \begin{itemize}
    \item $\sigma_{1}: [ \b{\tau} , \bar{\tau} ] \rightarrow [ \b{\tau} , \bar{\tau} ]$ (Report ex ante Typ $\tau$)
    \item $\sigma_{2}: [ \b{\tau} , \bar{\tau} ] \times [ \b{\theta} , \bar{\theta} ]
    \times [ \b{\tau} , \bar{\tau} ]  \rightarrow [ \b{\theta} , \bar{\theta} ]$ (Report ex post Typ $\theta$)
  \end{itemize}
\end{frame}

\subsection{Das Revelations-Prinzip}
\begin{frame}
  \frametitle{Das Revelations-Prinzip}
  \justifying
  \begin{thmP}[Nur Beweisidee und Interpretation]
    F"ur jeden dynamischen Mechanismus $\Gamma$ und jede optimale K"auferstrategie $\sigma$ in $\Gamma$ gibt es
    einen direkten Mechanismus $\Gamma'$ und eine optimale K"auferstrategie $\sigma =(\sigma_{1}', \sigma_{2}')$
    so dass gilt: \newline \newline
    i) Die Strategie $\sigma$ gen"ugt
    \begin{equation*}
      \sigma_{1}' (\tau) = \tau \ \forall  \tau \in [\b{\tau} , \bar{\tau} ]
      \quad \text{und} \quad
      \sigma_{2}' (\tau, \theta, \tau) = \theta \ \forall  \theta \in [\b{\theta} , \bar{\theta} ], \tau \in [\b{\tau} , \bar{\tau} ]
    \end{equation*}
    ii) Wenn der K"aufer seine optimale Strategie spielt, dann ist f"ur alle
    $(\tau, \theta) \in [\b{\theta} , \bar{\theta} ] \times [\b{\tau} , \bar{\tau} ]$ die
    Wahrscheinlichkeit $q( \tau, \theta)$ und die Auszahlung $t(\tau, \theta)$ unter $\Gamma'$ identisch mit der
    Wahrscheinlichkeit eines Kaufes und der erwarteten Auszahlung unter $\Gamma$.
  \end{thmP}
\end{frame}

\begin{frame}
  \frametitle{Folgerungen aus dem Revelations-Prinzip}
  \justifying
  Betrachte direkte Mechanismen:
  \begin{enumerate}
    \item Im Gleichgewicht: Die Wahrheit wird berichtet (keine Aussage für $\tau$ m"oglich?)
    \item Kein Gleichgewicht: Keine Aussage m"oglich
  \end{enumerate}
  $\Rightarrow$ Definiere \textcolor{blue}{Anreiz-kompatibel} und \textcolor{blue}{individuell rational}: \\
  \begin{align*}
    \theta^{r} &: [ \b{\theta} , \bar{\theta} ] \rightarrow [ \b{\theta} , \bar{\theta} ] &\text{Berichtsfunktion} \\
    u( \tau, \theta) &= \theta q( \tau, \theta) - t(\tau, \theta) &\text{Nutzenfunktion} \\
    U( \tau') := \hat{U}( \tau', \tau) &= \int_{\b{\theta}}^{\bar{\theta}} u( \tau', \hat{\theta}) f(\hat{\theta} \mid \tau) \ d \hat{\theta} & \text{Interpretation?}\\
  \end{align*}
\end{frame}

\subsection{Anreiz-Kompatibilit"at und indiviuelle Rationalit"at}
\begin{frame}
  \frametitle{Anreiz-kompatibel und individuell rational}
  \justifying
  \begin{thmD}
    Ein direkter Mechanismus ist \textcolor{blue}{Anreiz-kompatibel}, wenn \newline
    i) er Anreiz-kompatibel gegeben seinem Ex-Post Typen $\theta$ ist:
    \begin{equation*}
      u( \tau, \theta) \ge \theta q( \tau, \theta') - t( \tau, \theta') \quad \forall \tau \in [ \b{\tau} , \bar{\tau} ] \ \text{und} \ \theta, \theta' \in [ \b{\theta} , \bar{\theta} ]
    \end{equation*}
    ii) er Anreiz-kompatibel gegeben seinem Ex-Ante Typen $\tau$ ist:
    \begin{align*}
      &U(\tau) \ge \int_{\b{\theta}}^{\bar{\theta}} [ \hat{\theta} q(\tau', \theta^{r}(\hat{\theta})) - t( \tau', \theta^{r}(\hat{\theta})) ] f(\hat{\theta} \mid \tau) d \hat{\theta}  \\
      &\forall \tau, \tau' \in [ \b{\tau} , \bar{\tau} ] \ \text{und} \ \theta^{r}:[ \b{\theta} , \bar{\theta} ] \rightarrow [ \b{\theta} , \bar{\theta} ].
    \end{align*}
    Ein direkter Mechanismus ist \textcolor{blue}{individuel rational}, wenn
    \begin{equation*}
      U(\tau) \ge 0 \ \ \forall \tau \in [ \b{\tau} , \bar{\tau} ]
    \end{equation*}
  \end{thmD}

\end{frame}

\begin{frame}
  \frametitle{Vereinfache Anreiz-Kompatibilit"at I}
  \justifying
  \begin{thmP}[Nur Beweisidee und Interpretation]
    Ein direkter Mechanismus ist \textcolor{blue}{Anreiz-kompatibel} genau dann, wenn
    \begin{align*}
      &i) \quad u( \tau, \theta) \ge \theta q( \tau, \theta') - t( \tau, \theta') \quad \forall \tau \in [ \b{\tau} , \bar{\tau} ] \ \text{und} \ \theta, \theta' \in [ \b{\theta} , \bar{\theta} ], \\
      &ii) \quad U(\tau) \ge \hat{U}(\tau' \mit \tau) \quad \forall \tau, \tau' \in [ \b{\tau} , \bar{\tau} ].
    \end{align*}
  \end{thmP}
  Betrachte jetzt Monotonie-Kriterien $\rightarrow$ Allgemein gilt nicht:
  \begin{enumerate}
    \item Anreiz-Kompatibilit"at des Ex-Ante Typen $\tau$ impliziert nicht Monotonie der Allokationsregel. \\
    $\Rightarrow$ Wesentlicher Unterschied zu statischen Screening Model.
    \item Betrachte Anreiz-Kompatibilit"at des Ex-Post Typen $\theta$.
  \end{enumerate}
\end{frame}

\begin{frame}
  \frametitle{Vereinfache Anreiz-Kompatibilit"at II}
  \justifying
  \begin{thmP}[Nur Beweisidee und Interpretation]
    Ein direkter Mechanismus ist Anreiz-kompatibel gegeben seinem Ex-Post Typen $\theta$ genau dann, wenn \newline \newline
    i) F"ur alle Ex-Ante Typen $\tau$ die Funktion $q(\tau, \theta)$ steigend in $\theta$ ist, \\
    ii) F"ur alle Ex-Ante Typen $\tau$ und Ex-Post Typen $\theta$:
    \begin{equation*}
      \dfrac{\delta u(\tau, \theta)}{\delta \theta} = q(\tau, \theta),
    \end{equation*}
    iii) F"ur alle Ex-Ante Typen $\tau$ und Ex-Post Typen $\theta$:
    \begin{equation*}
      t(\tau, \theta) = t(\tau, \b{\theta}) + (\theta q(\tau, \theta) - \b{\theta} q(\tau, \b{\theta})) - \int_{\b{\theta}}^{\theta} q(\tau, \hat{\theta}) \ d \hat{\theta}.
    \end{equation*}
  \end{thmP}
\end{frame}

\begin{frame}
  \frametitle{Implikationen aus Anreiz-Kompatibilit"at I}
  \justifying
  \begin{thmP}[Beweis an der Tafel]
    Wenn ein direkter Mechanismus Anreiz-kompatibel ist, dann gilt f"ur alle Ex-Ante Typen $\tau$:
    \begin{align*}
      i)& \ U'(tau) = - \int_{\b{\theta}}^{\bar{\theta}} q(\tau, \hat{\theta}) \dfrac{\delta F(\hat{\theta} \mid \tau)}{\delta \tau} \ d \hat{\theta}, \\
      ii)& \ \int_{\b{\theta}}^{\bar{\theta}} t( \tau, \hat{\theta}) f( \hat{theta} \mid \tau) \ d \hat{\theta} = \int_{\b{\theta}}^{\bar{\theta}} \hat{\theta} q(\tau, \hat{\theta})
      f(\hat{\theta} \mid \tau) \ d \hat{\theta} \\
      &+ \int_{\b{\theta}}^{\bar{\theta}} [t(\b{\tau}, \hat{\theta}) - \hat{\theta} q(\b{\tau}, \hat{\theta})] f( \hat{theta} \mid \b{\tau}) \ d \hat{\theta} \\
      &+ \int_{\b{\tau}}^{\tau} \int_{\b{\theta}}^{\bar{\theta}} q(\hat{\tau}, \hat{\theta}) \dfrac{\delta F(\hat{\theta} \mid \hat{\tau})}{\delta \tau} \ d \hat{\theta} \ d \hat{\tau}.
    \end{align*}
  \end{thmP}
\end{frame}

\begin{frame}
  \frametitle{Implikationen aus Anreiz-Kompatibilit"at II}
  \justifying
  Wir haben festgestellt:
  \begin{enumerate}
    \item Vorigen Propositionen implizieren zwei Restriktionen an $t(\tau, \theta)$.
    \item F"uhren diese Restriktionen zu Widerspr"uchen?
    \item Nein! N"achste Proposition impliziert: Gegeben $q(\tau, \theta)$, dann wird $t(\tau, \theta)$ festgelegt
    durch $t(\b{\tau}, \b{\theta}$).
  \end{enumerate}
\end{frame}

\begin{frame}
  \frametitle{Implikationen aus Anreiz-Kompatibilit"at III}
  \justifying
  \begin{thmP}[Beweis an der Tafel]
    Wenn ein direkter Mechanismus Anreiz-kompatibel ist, dann gilt
    \begin{equation*}
      t(\tau, \theta) = t_{0}(\tau) + \theta q(\tau, \theta) - \int_{\b{\theta}}^{\theta} q(\tau, \hat{\theta}) \ d \hat{\theta},
    \end{equation*}
    mit
    \begin{align*}
      t_{0}(\tau) = t(\b{\tau},\b{\theta}) - \b{\theta} q(\b{\tau}, \b{\theta})
      + \int_{\b{\tau}}^{\tau} \int_{\b{\theta}}^{\bar{\theta}} q(\hat{\tau}, \hat{\theta}) \dfrac{\delta F(\hat{\theta} \mid \hat{\tau})}{\delta \tau} \ d \hat{\theta} \ d \hat{\tau} \\
      + \int_{\b{\theta}}^{\bar{\theta}} \int_{\b{\theta}}^{\hat{\theta}} [q(\tau, x) f(\hat{\theta} \mid \tau) - q(\b{\tau}, x) f(\hat{\theta} \mid \b{\tau})] \ d x \ d \hat{\theta}.
    \end{align*}
  \end{thmP}
\end{frame}

\begin{frame}
  \frametitle{Existenz eines Transferplanes f"ur Anreiz-Kompatibilit"at}
  \justifying
  Anfangsproblem: Monotonie-Kriterium versagt im dynamischen Kontext f"ur Anreiz-Kompatibilit"at! \\
  $\Rightarrow$ Wir ben"otigen eine zus"atzliche Existenz-Eigenschaft.
  \begin{thmP}[Beweisskizze Tafel?]
    Wenn $q(\tau, \theta)$ wachsend in $\tau$ und $\theta$ ist, dann existiert einn Transferplan $t(\tau, \theta)$, so dass der direkte Mechanismus
    Anreiz-kompatibel ist.
  \end{thmP}
\end{frame}

\begin{frame}
  \frametitle{Individuel-rational und Anreiz-kompatibel}
  \justifying
  \begin{thmP}[Nur m"undlicher Beweis]
    Ein Anreiz-kompatibler direkter Mechanismus ist individuel-rationaler genau dann, wenn
    \begin{equation*}
      U(\b{\tau}) \ge 0.
    \end{equation*}
  \end{thmP}
\end{frame}

\subsection{Optimaler Verkaufs-Mechanismus}
\begin{frame}
  \frametitle{Die erwartete Auszahlung}
  \justifying
  \begin{thmL}[An der Tafel - 4 Zeiler]
    Die erwartete Auszahlung f"ur den Verk"aufer ergibt sich zu
    \begin{align*}
      &\int_{\b{\tau}}^{\bar{\tau}} \int_{\b{\theta}}^{\bar{\theta}} [\hat{\theta} q(\hat{\tau}, \hat{\theta}) - u(\hat{\tau}, \hat{\theta})]
      f(\hat{\theta} \mid \hat{\tau} ) g(\hat{\tau}) \ d \hat{\theta} \ d \hat{\tau} \\
      &= \int_{\b{\tau}}^{\bar{\tau}} \int_{\b{\theta}}^{\bar{\theta}} \psi(\hat{\tau}, \hat{\theta}) q(\hat{\tau}, \hat{\theta})
      f(\hat{\theta} \mid \hat{\tau}) g(\hat{\tau}) \ d \hat{\theta} \ d \hat{\tau} - U(\b{\tau}),
    \end{align*}
    mit
    \begin{equation*}
      \psi(\tau, \theta) := \hat{\theta} + \dfrac{1-G(\hat{\tau})}{g(\hat{\tau})} \dfrac{ \delta F(\theta \mid \tau) / \delta \tau}{f(\theta \mid \tau)}.
    \end{equation*}
  \end{thmL}
\end{frame}

\begin{frame}
  \frametitle{Optimierung der erwarteten Auszahlung}
  \justifying
  \begin{thmA}
    $\psi(\tau, \theta)$ ist wachsend in $\tau$ und $\theta$.
  \end{thmA}
    Ferner gilt:
    \begin{enumerate}
      \item Maximiere erwartete Auszahlung unter Ber"ucksichtigung von individueler Rationalit"at
      $\Rightarrow U(\b{\tau}) = 0$
      \item Nach Modelannahme ist $\psi(\tau, \theta)$ stetig $\Rightarrow p(\tau) = \min \{\hat{\theta} \in [\b{\theta}, \bar{\theta}] \mid \psi(\tau, \hat{\theta}) \ge 0 \}$
      wohldefiniert
      \item Erwartete Auszahlung linear in $q(\tau, \theta)$
      $\Rightarrow q(\tau, \theta) = \left\{\begin{array}{lr}
        1, & \text{falls } \psi(\tau, \theta) \ge 0 \\
        0, & \text{sonst } 0
        \end{array} = \left\{\begin{array}{lr}
          1, & \text{falls } \theta \ge p(\tau) \\
          0, & \text{sonst } 0
          \end{array}$
    \end{enumerate}
\end{frame}

\begin{frame}
  \frametitle{Der optimale direkte Mechanismus}
  \justifying
  \begin{thmS}[Wenn noch Zeit, Beweis an Tafel]
    Unter der vorigen Annahme ist ein Anreiz-kompatibler und individuel rationaler direkter Mechanismus optimal genau dann, wenn
    \begin{equation*}
      q(\tau, \theta) = \left\{\begin{array}{lr}
        1, & \text{falls } \theta \ge p(\tau) \\
        0, & \text{sonst } 0
        \end{array}
    \end{equation*}
    und
    \begin{equation*}
      t(\tau, \theta) = \left\{\begin{array}{lr}
        t_{0}(\tau) p(\tau), & \text{falls } \theta \ge p(\tau) \\
        t_{0}, & \text{sonst } 0,
        \end{array}
    \end{equation*}
    mit $t_{0}$ wie in Proposition vorher. Ferner gilt
    \begin{equation*}
      t(b{\tau}, b{\theta}) = \hat{\theta} f(\hat{\theta} \mid \b{\tau}) \ d \hat{\theta} - p(\b{\tau}) [1 - F(p(\b{\tau}) \mid \b{\tau})] + \b{\theta} q(\b{\tau}, \b{\theta}).
    \end{equation*}
  \end{thmS}
\end{frame}

\section{Ausblicke}

\subsection{Die Rolle der privaten Information}
\begin{frame}
\frametitle{Modeltransformation}
\justifying
\begin{itemize}
  \item Modellerweiterung durch unabh"angiges Signal
  \begin{equation*}
    \gamma := F(\theta \mid \tau) \Leftrightarrow \theta = F^{-1}(\gamma \mid \tau)
  \end{equation*}
  \item $\gamma$ unabh"angig von $\tau$
\end{itemize}
\begin{thmL}
  Sei $\tilde{\psi}(\tau, \gamma) := \psi(\tau, F^{-1}(\gamma \mid \tau))$. Dann gilt
  \begin{equation*}
    \tilde{\psi}(\tau, \gamma) = F^{-1}(\gamma \mid \tau) - \dfrac{1-G(\tau)}{g(\tau)} \dfrac{\delta F^{-1}(\gamma \mid \tau)}{\delta \tau}.
  \end{equation*}
\end{thmL}
\end{frame}

\begin{frame}
\frametitle{Anreiz-kompatibel im Bezug auf $\gamma$}
\justifying
\begin{thmD}
  Wir nennen einen direkten Mechanismus $(\tilde{q}(\tau, \gamma), \tilde{t}(\tau, \gamma))$ \textcolor{blue}{Anreiz-kompatibel im Bezug auf $\gamma$}, wenn
  \begin{equation*}
    U(\tau) \geq \int_{0}^{1} F^{-1}(\gamma \mid \tau) \tilde{q}(\tau', \gamma) - \tilde{t}(\tau', \gamma) \ d \gamma \text{~ f"ur alle } \tau' \in [\b{\tau}, \bar{\tau}].
  \end{equation*}
\end{thmD}
\begin{thmP}
  Wenn der direkten Mechanismus $(\tilde{q}(\tau, \gamma), \tilde{t}(\tau, \gamma))$ Anreiz-kompatibel bez"uglich $\gamma$ ist, dann ist
  \begin{equation*}
    U(\tau) = U(\b{\tau}) + \int_{\b{\tau}}^{\tau} \int_{0}^{1} \dfrac{\delta F^{-1}(\gamma \mid \hat{\tau})}{\delta \tau} \tilde{q}(\hat{\tau}, \gamma) \ d \gamma \ d \hat{\tau}.
  \end{equation*}
\end{thmP}
\end{frame}

\begin{frame}
\frametitle{Optimalit"at im privaten und "offentlichen Fall}
\justifying
\begin{thmP}
  Angenommen $\psi(\tau, \theta)$ ist wachsend in $\tau$ und $\theta$. Wenn der direkten Mechanismus $(\tilde{q}(\tau, \gamma), \tilde{t}(\tau, \gamma))$ optimal ist bei privat bekanntem $\gamma$, so ist er
  auch optimal bei "offentlichem bekanntem $\gamma$.
\end{thmP}
\begin{itemize}
  \item Verk"aufer entzieht zus"atzliche private Informationen die nach Signal $\tau$ erfahren werden zu Kosten $0$
  \item Verk"aufer will so viel ex post private Informationen wie m"oglich entziehen
  \item Verk"aufer will so fr"uh wie m"oglich Mechanismus vorschlagen
\end{itemize}
\end{frame}

\subsection{Sequentielles Mechanismus Design}
\begin{frame}
\frametitle{Modellerweiterung auf $N$ K"aufer I}
\justifying
Betrachte indiziertes Modell
\begin{itemize}
  \item Spielermenge $I = \{ 1,...,N \}$
  \item Signal $\tau_{i}$ und Bewertung $\theta_{i}$ mit $i \in I$
  \item $\tau:= (\tau_{1}, ...,\tau_{N}), \theta := (\theta_{1}, ...,\theta_{N}), \Tau := [\b{\tau}, \bar{\tau}], \Theta := [\b{\theta}, \bar{\theta}]$
  \item $q(\tau, \theta) = (q_{1}(\tau, \theta), ...,q_{N}(\tau, \theta))$ und $\Delta := \{ (q_{1},...,q_{N}) \mid 0 \leq q_{i} \leq 1 \ \forall i \in I, \sum_{i \in I} q_{i} \leq 1 \}$
\end{itemize}
\begin{thmD}
  Ein (dynamischer) \textcolor{blue}{direkter Mechanismus} besteht aus den beiden Funktionen
  \begin{equation*}
    q: \Tau \times \Theta \rightarrow \Delta
    \quad \text{und} \quad
    t_{i}:\Tau \times \Theta \rightarrow \mathbb{R}.
  \end{equation*}
\end{thmD}
\end{frame}

\begin{frame}
\frametitle{Modellerweiterung auf $N$ K"aufer II}
\justifying
Indizierung setzt sich durch alle vorigen Ergebnisse fort:
\begin{itemize}
  \item Maximiere die erwartete Auszahlung des Verk"aufers
  \item Charakterisierung optimaler direkter Mechanismen $\{ q_{i}, t_{i}\}$
  \item Implementierung durch \textcolor{blue}{Benachteiligte Auktion}
  \begin{itemize}
    \item 1. Runde: M"oglichkeit von Abgabe $t_{0}(\tau_{i})$ und Zuteilung einer Pr"amie $p(\tau_{i})$
    \item 2. Runde: Zweitpreisauktion zuz"uglich $p(\tau_{i})$
    \item Wie ist die Abgabe $t_{0}(\tau_{i})$ und die Pr"amie $p(\tau_{i})$ zu w"ahlen?
  \end{itemize}
\end{itemize}
\end{frame}

\subsection{Dynamische Allokationen}
\begin{frame}
\frametitle{Wiederholtes Spiel mit Diskontierung}
\justifying
\begin{itemize}
  \item Zwei-Personen-Spiel, fixiere $\theta > 0$ und betrachte Perioden $\tau = 1,...,T$
  \item Modelliere mit Diskontierungsfaktor $\delta \in [0,1)$
  \item Betrachte periodenabh"angige $q_{\tau}$ und $t_{\tau}$
  \item $u(\theta) := \sum_{\tau=1}^{T} \delta^{\tau -1} (\theta q_\tau(\theta) - t_\tau(\theta))$
\end{itemize}

\begin{thmD}
  Ein (dynamischer) \textcolor{blue}{direkter Mechanismus} besteht aus den beiden Funktionen
  \begin{equation*}
    q:[ \b{\theta} , \bar{\theta} ] \rightarrow [0,1]^{T}
    \quad \text{und} \quad
    t:[ \b{\theta} , \bar{\theta} ] \rightarrow \mathbb{R}^{T}.
  \end{equation*}
\end{thmD}

\end{frame}

\begin{frame}
\frametitle{Anreiz-kompatibel und individuel-rational}
\justifying
\begin{thmD}
  Ein direkter Mechanismus bestehend aus den beiden Funktionen $q=(q_{1},...,q_{T})$ und $t = (t_{1}, ..., t_{T})$ ist \textcolor{blue}{Anreiz-kompatibel}, wenn f"ur alle $\theta,\theta' \in [ \b{\theta} , \bar{\theta} ]$
  \begin{equation*}
    u(\theta) \geq \sum_{\tau=1}^{T} \delta^{\tau -1} (\theta q_\tau(\theta') - t_\tau(\theta')).
  \end{equation*}
  Ein direkter Mechanismus ist \textcolor{blue}{individuel rational}, wenn
  \begin{equation*}
    u(\theta) \ge 0 \ \ \forall \theta \in [ \b{\theta} , \bar{\theta} ].
  \end{equation*}
\end{thmD}
\end{frame}

\begin{frame}
\frametitle{Optimalit"atskriterium}
\justifying
Die Optimalit"at l"asst sich charakterisieren:
\begin{itemize}
  \item Optimaler Mechanismus reduziert sich gem"a\ss \ Kapitel 2: \\ Eine Periode ohne Diskontierung
  \item K"aufer- vs. Verk"auferstrategien
  \item Wie kann ich gegenseitige Beeinflussung verhindern?
\end{itemize}
\end{frame}


\end{document}
