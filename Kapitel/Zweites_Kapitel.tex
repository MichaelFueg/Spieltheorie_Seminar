\section{Dynamische private Information}

\begin{frame}
  \frametitle{Mathemtische Modellierung}
  \justifying
  Sei im Folgenden f"ur das Signal $\tau$
  \begin{itemize}
    \item Kummulierte Verteilung $G(\tau)$
    \item Positive Dichte $g(\tau)$
    \item Tr"agermenge $[ \text{\b{\tau}} , \bar{\tau} ]$
  \end{itemize}
  Sei im Folgenden f"ur die Bewertung $\theta$
  \begin{itemize}
    \item Kummulierte Verteilung $F(\theta \mid \tau)$
    \item Korrespondierende Dichte $f(\theta \mid \tau)$
    \item Tr"agermenge $[ \text{\b{\theta}} , \bar{\theta} ]$ mit $0 \leq \text{\b{\theta}} < \bar{\theta} $ f"ur alle $\tau \in [ \text{\b{\tau}} , \bar{\tau} ]$
  \end{itemize}
\end{frame}

\begin{frame}
  \frametitle{Annahmen}
  \justifying
  Wir setzen im Folgenden voraus, dass
  \begin{itemize}
    \item Tr"agermenge von $F(\theta \mid \tau)$ ist $[ \b{\theta} , \bar{\theta} ]$ f"ur alle
    $\tau \in [ \b{\tau} , \bar{\tau} ], 0 \leq \b{\theta} < \bar{\theta}$
    \item Tr"agermenge von $F(\theta \mid \tau)$ unabh"angig von $\tau$
    \item $f(\theta \mid \tau) > 0$ f"ur alle $\tau \in [ \b{\tau} , \bar{\tau} ]$
    und $\theta \in [ \b{\theta} , \bar{\theta} ]$
    \item $F(\theta \mid \tau)$ und $f(\theta \mid \tau)$ sind stetig differenzierbar in $\tau$
    \item F"ur die Familie $F( \cdot \mid \tau)$ mit $\tau \in [ \b{\tau} , \bar{\tau} ]$ gilt
    \begin{equation}
      \tag{FOSD}
      \delta F(\theta \mid \tau)/ \delta \tau < 0 \ f"ur \ alle \ \theta \in  ( \b{\theta} , \bar{\theta} )
    \end{equation}
  \end{itemize}
\end{frame}

\subsection{Dynamischer direkter Mechanismus}
\begin{frame}
  \frametitle{Dynamischer direkter Mechanismus}
  \justifying
  \begin{thmD}
    Ein (dynamischer) \textcolor{blue}{direkter Mechanismus} besteht aus den beiden Funktionen
    \begin{equation*}
      q:[ \b{\tau} , \bar{\tau} ] \times [ \b{\theta} , \bar{\theta} ] \rightarrow [0,1]
      \quad \text{und} \quad
      t:[ \b{\tau} , \bar{\tau} ] \times [ \b{\theta} , \bar{\theta} ] \rightarrow \mathbb{R}.
    \end{equation*}
  \end{thmD}
  Zwei wesentliche Haupt"anderungen:
  \begin{enumerate}
    \item Verwende nun das kartesische Produkt $[ \b{\tau} , \bar{\tau} ] \times [ \b{\theta} , \bar{\theta} ]$
    \item Zwei Report-Ebenen: $\tau$ und $\theta$
  \end{enumerate}
\end{frame}

\begin{frame}
  \frametitle{Ein-Personen-Entscheidungsproblem}
  \justifying
  Optimale Entscheidungsregel ist das Paar $\sigma = (\sigma_{1}, \sigma_{2})$ mit
  \begin{itemize}
    \item $\sigma_{1}: [ \b{\tau} , \bar{\tau} ] \rightarrow [ \b{\tau} , \bar{\tau} ]$ (Report ex ante Typ $\tau$)
    \item $\sigma_{2}: [ \b{\tau} , \bar{\tau} ] \times [ \b{\theta} , \bar{\theta} ]
    \times [ \b{\tau} , \bar{\tau} ]  \rightarrow [ \b{\theta} , \bar{\theta} ]$ (Report ex post Typ $\theta$)
  \end{itemize}
\end{frame}

\subsection{Das Revelations-Prinzip}
\begin{frame}
  \frametitle{Das dynamische Revelations-Prinzip}
  \justifying
  \begin{thmP}
    F"ur jeden dynamischen Mechanismus $\Gamma$ und jede optimale K"auferstrategie $\sigma$ in $\Gamma$ gibt es
    einen direkten Mechanismus $\Gamma'$ und eine optimale K"auferstrategie $\sigma' =(\sigma_{1}', \sigma_{2}')$, so dass gilt: \newline \newline
    i) Die Strategie $\sigma'$ gen"ugt
    \begin{equation*}
      \sigma_{1}' (\tau) = \tau \ \forall  \tau \in [\b{\tau} , \bar{\tau} ]
      \quad \text{und} \quad
      \sigma_{2}' (\tau, \theta, \tau) = \theta \ \forall  \theta \in [\b{\theta} , \bar{\theta} ], \tau \in [\b{\tau} , \bar{\tau} ].
    \end{equation*}
    ii) Wenn der K"aufer seine optimale Strategie spielt, dann ist f"ur alle
    $(\tau, \theta) \in [\b{\tau} , \bar{\tau} ] \times [\b{\theta} , \bar{\theta} ] $ die
    Wahrscheinlichkeit $q( \tau, \theta)$ und die Auszahlung $t(\tau, \theta)$ unter $\Gamma'$ identisch mit der
    Wahrscheinlichkeit eines Kaufes und der erwarteten Auszahlung unter $\Gamma$.
  \end{thmP}
\end{frame}

\begin{frame}
  \frametitle{Folgerungen aus dem Revelations-Prinzip}
  \justifying
  Betrachte direkte Mechanismen:
  \begin{enumerate}
    \item Im Gleichgewicht: Die Wahrheit wird berichtet
    \item Kein Gleichgewicht: Keine Aussage m"oglich
  \end{enumerate}
  \begin{align*}
    \theta^{r} &: [ \b{\theta} , \bar{\theta} ] \rightarrow [ \b{\theta} , \bar{\theta} ] \\
    u( \tau, \theta) &= \theta q( \tau, \theta) - t(\tau, \theta) \\
    \hat{U}( \tau' | \tau) &= \int_{\text{\b{\theta}}}^{\bar{\theta}} u( \tau', \hat{\theta}) f(\hat{\theta} \mid \tau) \ d \hat{\theta} \\
    U(\hat{\tau}) &= \hat{U}( \hat{\tau} | \hat{\tau})
  \end{align*}
\end{frame}

\subsection{Anreiz-Kompatibilit"at und indiviuelle Rationalit"at}
\begin{frame}
  \frametitle{Anreiz-kompatibel und individuell rational}
  \justifying
  \begin{thmD}
    Ein direkter Mechanismus ist \textcolor{blue}{Anreiz-kompatibel}, wenn \newline
    i) er Anreiz-kompatibel im Bezug auf seinem ex post Typen $\theta$ ist:
    \begin{equation*}
      u( \tau, \theta) \ge \theta q( \tau, \theta') - t( \tau, \theta') \quad \forall \tau \in [ \b{\tau} , \bar{\tau} ] \ \text{und} \ \theta, \theta' \in [ \b{\theta} , \bar{\theta} ].
    \end{equation*}
    ii) er Anreiz-kompatibel im Bezug auf seinem ex ante Typen $\tau$ ist:
    \begin{align*}
      &U(\tau) \ge \int_{\text{\b{\theta}}}^{\bar{\theta}} [ \hat{\theta} q(\tau', \theta^{r}(\hat{\theta})) - t( \tau', \theta^{r}(\hat{\theta})) ] f(\hat{\theta} \mid \tau) d \hat{\theta}  \\
      &\forall \tau, \tau' \in [ \b{\tau} , \bar{\tau} ] \ \text{und} \ \theta^{r}:[ \b{\theta} , \bar{\theta} ] \rightarrow [ \b{\theta} , \bar{\theta} ].
    \end{align*}
    Ein direkter Mechanismus ist \textcolor{blue}{individuel rational}, wenn
    \begin{equation*}
      U(\tau) \ge 0 \ \ \forall \tau \in [ \b{\tau} , \bar{\tau} ].
    \end{equation*}
  \end{thmD}
\end{frame}

\begin{frame}
  \frametitle{Vereinfache Anreiz-Kompatibilit"at I}
  \justifying
  \begin{thmP}
    Ein direkter Mechanismus ist \textcolor{blue}{Anreiz-kompatibel} genau dann, wenn
    \begin{align*}
      &i) \quad u( \tau, \theta) \ge \theta q( \tau, \theta') - t( \tau, \theta') \quad \forall \tau \in [ \b{\tau} , \bar{\tau} ] \ \text{und} \ \theta, \theta' \in [ \b{\theta} , \bar{\theta} ], \\
      &ii) \quad U(\tau) \ge \hat{U}(\tau' \mid \tau) \quad \forall \tau, \tau' \in [ \b{\tau} , \bar{\tau} ].
    \end{align*}
  \end{thmP}
  Betrachte jetzt Monotonie-Kriterien $\rightarrow$ gelten im Allgemeinen nicht:
  \begin{enumerate}
    \item Anreiz-Kompatibilit"at des ex ante Typen $\tau$ impliziert nicht Monotonie der Allokationsregel. \\
    $\Rightarrow$ Wesentlicher Unterschied zu statischem Screening Model.
    \item Betrachte Anreiz-Kompatibilit"at gegeben des ex post Typen $\theta$.
  \end{enumerate}
\end{frame}

\begin{frame}
  \frametitle{Vereinfache Anreiz-Kompatibilit"at II}
  \justifying
  \begin{thmP}
    Ein direkter Mechanismus ist Anreiz-kompatibel gegeben seinem ex post Typen $\theta$ genau dann, wenn \newline \newline
    i) f"ur alle ex ante Typen $\tau$ die Funktion $q(\tau, \theta)$ steigend in $\theta$ ist, \\
    ii) f"ur alle ex ante Typen $\tau$ und ex post Typen $\theta$:
    \begin{equation*}
      \dfrac{\delta u(\tau, \theta)}{\delta \theta} = q(\tau, \theta),
    \end{equation*}
    iii) f"ur alle ex ante Typen $\tau$ und ex post Typen $\theta$:
    \begin{equation*}
      t(\tau, \theta) = t(\tau, \b{\theta}) + (\theta q(\tau, \theta) - \b{\theta} q(\tau, \b{\theta})) - \int_{\text{\b{\theta}}}^{\theta} q(\tau, \hat{\theta}) \ d \hat{\theta}.
    \end{equation*}
  \end{thmP}
\end{frame}

\begin{frame}
  \frametitle{Beispiel f"ur Anreiz-Kompatibilit"at}
  \justifying
  Wir definieren zun"achst $[\b{\theta}, \bar{\theta}] = [0, \bar{\theta}]$ und $[\b{\tau}, \bar{\tau}] = [0, \bar{\tau}]$.
  \begin{thmL}
    Der direkte Mechanismus charakterisiert durch
    \begin{equation*}
      q(\tau, \theta) := 1- e^{-(\tau + \theta)} \text{ und } t(\tau, \theta) := - e^{-(\tau + \theta)} \cdot (1 + \theta)
    \end{equation*}
    ist Anreiz-kompatibel.
  \end{thmL}
\end{frame}

\begin{frame}
  \frametitle{Implikationen aus Anreiz-Kompatibilit"at I}
  \justifying
  \begin{thmP}
    Wenn ein direkter Mechanismus Anreiz-kompatibel ist, dann gilt f"ur alle ex ante Typen $\tau$:
    \begin{align*}
      i)& \ U'(\tau) = - \int_{\text{\b{\theta}}}^{\bar{\theta}} q(\tau, \hat{\theta}) \dfrac{\delta F(\hat{\theta} \mid \tau)}{\delta \tau} \ d \hat{\theta}, \\
      ii)& \ \int_{\text{\b{\theta}}}^{\bar{\theta}} t( \tau, \hat{\theta}) f( \hat{\theta} \mid \tau) \ d \hat{\theta} = \int_{\text{\b{\theta}}}^{\bar{\theta}} \hat{\theta} q(\tau, \hat{\theta})
      f(\hat{\theta} \mid \tau) \ d \hat{\theta} \\
      &+ \int_{\text{\b{\theta}}}^{\bar{\theta}} [t(\b{\tau}, \hat{\theta}) - \hat{\theta} q(\b{\tau}, \hat{\theta})] f( \hat{\theta} \mid \b{\tau}) \ d \hat{\theta} \\
      &+ \int_{\b{\tau}}^{\tau} \int_{\text{\b{\theta}}}^{\bar{\theta}} q(\hat{\tau}, \hat{\theta}) \dfrac{\delta F(\hat{\theta} \mid \hat{\tau})}{\delta \tau} \ d \hat{\theta} \ d \hat{\tau}.
    \end{align*}
  \end{thmP}
\end{frame}

\bgroup
\setbeamercolor{background canvas}{bg=black}
\begin{frame}[plain]{}
\end{frame}
\egroup

\begin{frame}
  \frametitle{Implikationen aus Anreiz-Kompatibilit"at II}
  \justifying
  Wir haben festgestellt:
  \begin{enumerate}
    \item Vorige Propositionen implizieren zwei Restriktionen an $t(\tau, \theta)$.
    \item N"achste Proposition impliziert: Gegeben $q(\tau, \theta)$, dann wird $t(\tau, \theta)$ festgelegt
    durch $t(\b{\tau}, \b{\theta}$).
  \end{enumerate}
\end{frame}

\begin{frame}
  \frametitle{Implikationen aus Anreiz-Kompatibilit"at III}
  \justifying
  \begin{thmP}
    Wenn ein direkter Mechanismus Anreiz-kompatibel ist, dann gilt
    \begin{equation*}
      t(\tau, \theta) = t_{0}(\tau) + \theta q(\tau, \theta) - \int_{\text{\b{\theta}}}^{\theta} q(\tau, \hat{\theta}) \ d \hat{\theta},
    \end{equation*}
    mit
    \begin{align*}
      t_{0}(\tau) = t(\b{\tau},\b{\theta}) - \b{\theta} q(\b{\tau}, \b{\theta})
      + \int_{\b{\tau}}^{\tau} \int_{\text{\b{\theta}}}^{\bar{\theta}} q(\hat{\tau}, \hat{\theta}) \dfrac{\delta F(\hat{\theta} \mid \hat{\tau})}{\delta \tau} \ d \hat{\theta} \ d \hat{\tau} \\
      + \int_{\text{\b{\theta}}}^{\bar{\theta}} \int_{\text{\b{\theta}}}^{\hat{\theta}} [q(\tau, x) f(\hat{\theta} \mid \tau) - q(\b{\tau}, x) f(\hat{\theta} \mid \b{\tau})] \ d x \ d \hat{\theta}.
    \end{align*}
  \end{thmP}
\end{frame}

\begin{frame}
  \frametitle{Existenz eines Transferplanes f"ur Anreiz-Kompatibilit"at}
  \justifying
  Anfangsproblem: Monotonie-Kriterium versagt im dynamischen Kontext f"ur Anreiz-Kompatibilit"at! \\
  \begin{thmP}
    Wenn $q(\tau, \theta)$ wachsend in $\tau$ und $\theta$ ist, dann existiert ein Transferplan $t(\tau, \theta)$, so dass der direkte Mechanismus
    Anreiz-kompatibel ist.
  \end{thmP}
\end{frame}

\begin{frame}
  \frametitle{Individuel-rational}
  \justifying
  \begin{thmP}
    Ein Anreiz-kompatibler direkter Mechanismus ist individuel-rational genau dann, wenn
    \begin{equation*}
      U(\b{\tau}) \ge 0.
    \end{equation*}
  \end{thmP}
\end{frame}

\begin{frame}
  \frametitle{Beispiel: Individuel-rational}
  \justifying
  \begin{thmL}
    Sei $[\b{\theta}, \bar{\theta}] = [0, \bar{\theta}]$ und $[\b{\tau}, \bar{\tau}] = [0, \bar{\tau}]$.
    Dann ist der Anreiz-kompatible direkte Mechanismus charakterisiert durch
      \begin{equation*}
        q(\tau, \theta) := 1- e^{-(\tau + \theta)} \text{ und } t(\tau, \theta) := - e^{-(\tau + \theta)} \cdot (1 + \theta)
      \end{equation*}
      individuel-rational.
  \end{thmL}
\end{frame}

\subsection{Optimaler Verkaufs-Mechanismus}
\begin{frame}
  \frametitle{Die erwartete Auszahlung}
  \justifying
  \begin{thmL}
    Die erwartete Auszahlung f"ur den Verk"aufer ergibt sich als
    \begin{align*}
      &\int_{\b{\tau}}^{\bar{\tau}} \int_{\text{\b{\theta}}}^{\bar{\theta}} [\hat{\theta} q(\hat{\tau}, \hat{\theta}) - u(\hat{\tau}, \hat{\theta})]
      f(\hat{\theta} \mid \hat{\tau} ) g(\hat{\tau}) \ d \hat{\theta} \ d \hat{\tau} \\
      &= \int_{\b{\tau}}^{\bar{\tau}} \int_{\text{\b{\theta}}}^{\bar{\theta}} \psi(\hat{\tau}, \hat{\theta}) q(\hat{\tau}, \hat{\theta})
      f(\hat{\theta} \mid \hat{\tau}) g(\hat{\tau}) \ d \hat{\theta} \ d \hat{\tau} - U(\b{\tau}),
    \end{align*}
    mit
    \begin{equation*}
      \psi(\tau, \theta) := \hat{\theta} + \dfrac{1-G(\hat{\tau})}{g(\hat{\tau})} \dfrac{ \delta F(\theta \mid \tau) / \delta \tau}{f(\theta \mid \tau)}.
    \end{equation*}
  \end{thmL}
\end{frame}

\begin{frame}
  \frametitle{Optimierung der erwarteten Auszahlung}
  \justifying
  \begin{thmA}
    $\psi(\tau, \theta)$ ist wachsend in $\tau$ und $\theta$.
  \end{thmA}
    Ferner gilt:
    \begin{enumerate}
      \item Maximiere erwartete Auszahlung unter Ber"ucksichtigung von individueler Rationalit"at
      $\Rightarrow U(\b{\tau}) = 0$
      \item Nach Modellannahme ist $\psi(\tau, \theta)$ stetig $\Rightarrow p(\tau) = \min \{\hat{\theta} \in [\b{\theta}, \bar{\theta}] \mid \psi(\tau, \hat{\theta}) \ge 0 \}$
      wohldefiniert
      \item Erwartete Auszahlung linear in $q(\tau, \theta)$
      $\Rightarrow q(\tau, \theta) = \left\{\begin{array}{lr}
        1, & \text{falls } \psi(\tau, \theta) \ge 0 \\
        0, & \text{sonst }
        \end{array} = \left\{\begin{array}{lr}
          1, & \text{falls } \theta \ge p(\tau) \\
          0, & \text{sonst }
          \end{array}$
    \end{enumerate}
\end{frame}

\begin{frame}
  \frametitle{Der optimale direkte Mechanismus}
  \justifying
  \begin{thmS}
    Unter der vorigen Annahme ist ein Anreiz-kompatibler und individuel rationaler direkter Mechanismus optimal genau dann, wenn
    \begin{equation*}
      q(\tau, \theta) = \left\{\begin{array}{lr}
        1, & \text{falls } \theta \ge p(\tau) \\
        0, & \text{sonst }
        \end{array}
    \end{equation*}
    und
    \begin{equation*}
      t(\tau, \theta) = \left\{\begin{array}{lr}
        t_{0}(\tau) + p(\tau), & \text{falls } \theta \ge p(\tau) \\
        t_{0}(\tau), & \text{sonst },
        \end{array}
    \end{equation*}
    mit $t_{0}$ wie in Proposition vorher. Ferner gilt
    \begin{equation*}
      t(\b{\tau}, \b{\theta}) = \int_{p(\b{\tau})}^{\bar{\theta}}\hat{\theta} f(\hat{\theta} \mid \b{\tau}) \ d \hat{\theta} - p(\b{\tau}) [1 - F(p(\b{\tau}) \mid \b{\tau})] + \b{\theta} q(\b{\tau}, \b{\theta}).
    \end{equation*}
  \end{thmS}
\end{frame}

\bgroup
\setbeamercolor{background canvas}{bg=black}
\begin{frame}[plain]{}
\end{frame}
\egroup
