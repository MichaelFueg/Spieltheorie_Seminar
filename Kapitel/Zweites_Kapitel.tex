\section{Dynamische private Informationen}

\begin{frame}
  \frametitle{Modelrahmen}
  \justifying
  Modellrahmen ist Zwei-Personen Spiel:
  \begin{itemize}
    \item \textcolor{blue}{Verk"aufer}
    \begin{itemize}
      \item Verkauft unteilbares Gut
      \item Slegt Mechanismus $\Gamma$ fest
    \end{itemize}
    \item \textcolor{blue}{K"aufer}
    \begin{itemize}
      \item Bewertet Gut durch $\theta > 0$
      \item $\theta$ erst nach Aktzeptieren des Mechanismus $\Gamma$ bekannt
      \item Erh"alt aber vorher Signal $\tau$, welches mit $\theta$ korreliert ist
    \end{itemize}
  \end{itemize}
\end{frame}

\begin{frame}
  \frametitle{Mathemtische Modellierung}
  \justifying
  Sei im Folgenden f"ur das Signal $\tau$
  \begin{itemize}
    \item Kommulierte Verteilung $G(\tau)$
    \item Positive Dichte $g(\tau)$
    \item Tr"agermenge $[ \b{\tau} , \bar{\tau} ]$
  \end{itemize}
  Sei im Folgenden f"ur die Bewertung $\theta$
  \begin{itemize}
    \item Kommulierte Verteilung $F(\theta \mid \tau)$
    \item Korrespondierende Dichte $f(\theta \mid \tau)$
    \item Tr"agermenge $[ \b{\theta} , \bar{\theta} ]$ mit $0 \leq \b{\theta} < \bar{\theta} $ f"ur alle $\tau \in [ \b{\tau} , \bar{\tau} ]$
  \end{itemize}
\end{frame}

\begin{frame}
  \frametitle{Annahmen}
  \justifying
  Wir setzen im Folgenden voraus, dass
  \begin{itemize}
    \item Tr"agermenge von $F(\theta \mid \tau)$ ist $[ \b{\theta} , \bar{\theta} ]$ f"ur alle
    $\tau \in [ \b{\tau} , \bar{\tau} ], 0 \leq \b{\theta} < \bar{\theta}$
    \item Tr"agermenge von $F(\theta \mid \tau)$ unabh"angig von $\tau$
    \item $f(\theta \mid \tau) > 0$ f"ur alle $\tau \in [ \b{\tau} , \bar{\tau} ]$
    und $\theta \in [ \b{\theta} , \bar{\theta} ]$
    \item $F(\theta \mid \tau)$ und $f(\theta \mid \tau)$ sind stetig differenzierbar in $\tau$
    \item F"ur die Familie $F( \cdot \mid \tau)$ mit $\tau \in [ \b{\tau} , \bar{\tau} ]$ gilt
    \begin{equation}
      \tag{FOSD}
      \delta F(\theta \mid \tau)/ \delta \tau < 0 \ f"ur \ alle \ \theta \in [ \b{\theta} , \bar{\theta} ]
    \end{equation}
  \end{itemize}
\end{frame}

\begin{frame}
  \frametitle{Dynamischer direkter Mechanismus}
  \justifying
  \begin{thmD}
    Ein \textcolor{blue}{dynamischer direkter Mechanismus} besteht aus den beiden Funktionen
    \begin{equation*}
      q:[ \b{\tau} , \bar{\tau} ] \times [ \b{\theta} , \bar{\theta} ] \rightarrow [0,1]
      \quad \text{und} \quad
      t:[ \b{\tau} , \bar{\tau} ] \times [ \b{\theta} , \bar{\theta} ] \rightarrow \mathbb{R}.
    \end{equation*}
  \end{thmD}
  Zwei wesentliche Haupt"anderungen:
  \begin{enumerate}
    \item Verwende nun das kartesische Produkt $[ \b{\tau} , \bar{\tau} ] \times [ \b{\theta} , \bar{\theta} ]$
    \item ?
  \end{enumerate}
\end{frame}

\begin{frame}
  \frametitle{Ein-Personen-Entscheidungsproblem}
  \justifying
  Optimale Entscheidung ist das Paar $\sigma = (\sigma_{1}, \sigma_{2})$ mit
  \begin{itemize}
    \item $\sigma_{1}: [ \b{\tau} , \bar{\tau} ] \rightarrow [ \b{\tau} , \bar{\tau} ]$ (Report ex ante Typ $\tau$)
    \item $\sigma_{2}: [ \b{\tau} , \bar{\tau} ] \times [ \b{\theta} , \bar{\theta} ]
    \times [ \b{\tau} , \bar{\tau} ]  \rightarrow [ \b{\theta} , \bar{\theta} ]$ (Report ex post Typ $\theta$)
  \end{itemize}
\end{frame}

\begin{frame}
  \frametitle{?}
  \begin{thmP}
    F"ur jeden dynamischen Mechanismus $\Gamma$ und jede optimale K"auferstrategie $\sigma$ in $\Gamma$ gibt es
    einen direkten Mechanismus $\Gamma'$ und eine optimale K"auferstrategie $\sigma =(\sigma_{1}', \sigma_{2}')$
    so dass gilt: \newline \newline
    i) Die Strategie $\sigma$ gen"ugt
    \begin{equation*}
      \sigma_{1}' (\tau) = \tau \ \forall  \tau \in [\b{\tau} , \bar{\tau} ]
      \quad \text{und} \quad
      \sigma_{2}' (\tau, \theta, \tau) = \theta \ \forall  \theta \in [\b{\theta} , \bar{\theta} ], \tau \in [\b{\tau} , \bar{\tau} ]
    \end{equation*}
    ii) Wenn der K"aufer seine optimale Strategie spielt, dann ist f"ur alle
    $(\tau, \theta) \in [\b{\theta} , \bar{\theta} ] \times [\b{\tau} , \bar{\tau} ]$ die
    Wahrscheinlichkeit $q( \tau, \theta)$ und die Auszahlung $t(\tau, \theta)$ unter $\Gamma'$ identisch mit der
    Wahrscheinlichkeit eines Kaufes und der erwarteten Auszahlung unter $\Gamma$.
  \end{thmP}
\end{frame}
