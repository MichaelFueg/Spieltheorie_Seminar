\section{Ausblicke}

\subsection{Die Rolle der privaten Information}
\begin{frame}
\frametitle{Modeltransformation}
\justifying
\begin{itemize}
  \item Modellerweiterung durch unabh"angiges Signal
  \begin{equation*}
    \gamma := F(\theta \mid \tau) \Leftrightarrow \theta = F^{-1}(\gamma \mid \tau)
  \end{equation*}
  \item $\gamma$ unabh"angig von $\tau$
\end{itemize}
\begin{thmL}
  Sei $\tilde{\psi}(\tau, \gamma) := \psi(\tau, F^{-1}(\gamma \mid \tau))$. Dann gilt
  \begin{equation*}
    \tilde{\psi}(\tau, \gamma) = F^{-1}(\gamma \mid \tau) - \dfrac{1-G(\tau)}{g(\tau)} \dfrac{\delta F^{-1}(\gamma \mid \tau)}{\delta \tau}.
  \end{equation*}
\end{thmL}
\end{frame}

\begin{frame}
\frametitle{Anreiz-kompatibel bez"uglich $\gamma$}
\justifying
\begin{thmD}
  Wir nennen einen direkten Mechanismus $(\tilde{q}(\tau, \gamma), \tilde{t}(\tau, \gamma))$ \textcolor{blue}{Anreiz-kompatibel bez"uglich $\gamma$}, wenn
  \begin{equation*}
    U(\tau) \geq \int_{0}^{1} F^{-1}(\gamma \mid \tau) \tilde{q}(\tau', \gamma) - \tilde{t}(\tau', \gamma) \ d \gamma \text{~ f"ur alle } \tau' \in [\b{\tau}, \bar{\tau}].
  \end{equation*}
\end{thmD}
\begin{thmP}
  Wenn der direkten Mechanismus $(\tilde{q}(\tau, \gamma), \tilde{t}(\tau, \gamma))$ Anreiz-kompatibel bez"uglich $\gamma$ ist, dann ist
  \begin{equation*}
    U(\tau) = U(\b{\tau}) + \int_{\b{\tau}}^{\tau} \int_{0}^{1} \dfrac{\delta F^{-1}(\gamma \mid \hat{\tau})}{\delta \tau} \tilde{q}(\hat{\tau}, \gamma) \ d \gamma \ d \hat{\tau}.
  \end{equation*}
\end{thmP}
\end{frame}

\begin{frame}
\frametitle{Optimalit"at im privaten und "offentlichen Fall}
\justifying
\begin{thmP}
  Angenommen $\psi(\tau, \theta)$ ist wachsend in $\tau$ und $\theta$. Wenn der direkten Mechanismus $(\tilde{q}(\tau, \gamma), \tilde{t}(\tau, \gamma))$ optimal ist bei privat bekanntem $\gamma$, so ist er
  auch optimal bei "offentlichem bekanntem $\gamma$.
\end{thmP}
\begin{itemize}
  \item Verk"aufer entzieht zus"atzliche private Informationen die nach Signal $\tau$ erfahren werden zu Kosten $0$
  \item Verk"aufer will so viel ex post private Informationen wie m"oglich entziehen
  \item Verk"aufer will so fr"uh wie m"oglich Mechanismus vorschlagen
\end{itemize}
\end{frame}

\subsection{Sequentielles Mechanismus Design}
\begin{frame}
\frametitle{Modellerweiterung auf $N$ K"aufer I}
\justifying
Betrachte indiziertes Modell
\begin{itemize}
  \item Spielermenge $I = \{ 1,...,N \}$
  \item Signal $\tau_{i}$ und Bewertung $\theta_{i}$ mit $i \in I$
  \item $\tau:= (\tau_{1}, ...,\tau_{N}), \theta := (\theta_{1}, ...,\theta_{N}), \Tau := [\b{\tau}, \bar{\tau}], \Theta := [\b{\theta}, \bar{\theta}]$
  \item $q(\tau, \theta) = (q_{1}(\tau, \theta), ...,q_{N}(\tau, \theta))$ und $\Delta := \{ (q_{1},...,q_{N}) \mid 0 \leq q_{i} \leq 1 \ \forall i \in I, \sum_{i \in I} q_{i} \leq 1 \}$
\end{itemize}
\begin{thmD}
  Ein (dynamischer) \textcolor{blue}{direkter Mechanismus} besteht aus den beiden Funktionen
  \begin{equation*}
    q: \Tau \times \Theta \rightarrow \Delta
    \quad \text{und} \quad
    t_{i}:\Tau \times \Theta \rightarrow \mathbb{R}.
  \end{equation*}
\end{thmD}
\end{frame}

\begin{frame}
\frametitle{Modellerweiterung auf $N$ K"aufer II}
\justifying
Indizierung setzt sich durch alle vorigen Ergebnisse fort:
\begin{itemize}
  \item Maximiere die erwartete Auszahlung des Verk"aufers
  \item Charakterisierung optimaler direkter Mechanismen $\{ q_{i}, t_{i}\}$
  \item Implementierung durch \textcolor{blue}{Benachteiligte Auktion}
  \begin{itemize}
    \item 1. Runde: M"oglichkeit von Abgabe $t_{0}(\tau_{i})$ und Zuteilung einer Pr"amie $p(\tau_{i})$
    \item 2. Runde: Zweitpreisauktion zuz"uglich $p(\tau_{i})$
    \item Wie ist die Abgabe $t_{0}(\tau_{i})$ und die Pr"amie $p(\tau_{i})$ zu w"ahlen?
  \end{itemize}
\end{itemize}
\end{frame}

\subsection{Dynamische Allokationen}
\begin{frame}
\frametitle{Wiederholtes spiel mit Diskontierung}
\justifying
\begin{itemize}
  \item Zweipersonenspiel, fixiere $\theta > 0$ und betrachte Perioden $\tau = 1,...,T$
  \item Modelliere mit Diskontierungsfaktor $\delta \in [0,1)$
  \item Betrachte periodenabh"angige $q_{\tau}$ und $t_{\tau}$
  \item $u(\theta) := \sum_{\tau=1}^{T} \delta^{\tau -1} (\theta q_\tau(\theta) - t_\tau(\theta))$
\end{itemize}

\begin{thmD}
  Ein (dynamischer) \textcolor{blue}{direkter Mechanismus} besteht aus den beiden Funktionen
  \begin{equation*}
    q:[ \b{\theta} , \bar{\theta} ] \rightarrow [0,1]^{T}
    \quad \text{und} \quad
    t:[ \b{\theta} , \bar{\theta} ] \rightarrow \mathbb{R}^{T}.
  \end{equation*}
\end{thmD}

\end{frame}

\begin{frame}
\frametitle{Anreiz-kompatibel und individuel-rational}
\justifying
\begin{thmD}
  Ein direkter Mechanismus bestehend aus den beiden Funktionen $q=(q_{1},...,q_{T})$ und $t = (t_{1}, ..., t_{T})$ ist \textcolor{blue}{Anreiz-kompatibel}, wenn f"ur alle $\theta,\theta' \in [ \b{\theta} , \bar{\theta} ]$
  \begin{equation*}
    u(\theta) \geq \sum_{\tau=1}^{T} \delta^{\tau -1} (\theta q_\tau(\theta') - t_\tau(\theta')).
  \end{equation*}
  Ein direkter Mechanismus ist \textcolor{blue}{individuel rational}, wenn
  \begin{equation*}
    u(\theta) \ge 0 \ \ \forall \theta \in [ \b{\theta} , \bar{\theta} ].
  \end{equation*}
\end{thmD}
\end{frame}

\begin{frame}
\frametitle{Optimalit"atskriterium}
\justifying
Die Optimalit"at l"asst sich charakterisieren:
\begin{itemize}
  \item Optimaler Mechanismus reduziert sich gem"a\ss \ Kapitel 2: \\ Eine Periode ohne Diskontierung
  \item K"aufer- vs. Verk"auferstrategien
  \item Wie kann ich gegenseitige Beeinflussung verhindern?
\end{itemize}
\end{frame}
